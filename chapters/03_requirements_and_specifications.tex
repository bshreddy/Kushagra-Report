\documentclass[../Report.tex]{subfiles}

\begin{document}

\chapter{Requirements and  Specifications}

\section{Software Requirements}

\subsection{Functional Requirements}

Following are the functional Requirements associated with our project:

1. User can login to the app using their respective Google account

2. User should be able to take an image of the app and query the server to get the prediction of the crop or disease

3. User should be able to view details of the crop or disease of the crop

4. User should be able to send the crop details or disease details to other users in a PDF Format

\subsection{Non-Functional Requirements}
\begin{description}
  \item[Performance: ] With the help of optimization techniques used in the app, accurate results are provided to users while performing tasks
  
  \item[Scalabilty: ] This application maintains a positive user experience for over thousands of users, and since the data is processed
at the server, Traffic surge can be handled in a simple manner

  \item[Usability: ] With the help of localization support in the application, users can perform necessary actions in their preferred local language

  \item[Availability: ] Since Android Platform has a Lion's share in mobile market we can easily reach and serve large number of users

  \item[Security: ] //TODO

  \item[Cost and Maintainability: ] //TODO
\end{description} 

\section{System Specifications}

\subsection{Software Specifications}

\begin{description}
  \item[Pytorch: ] Pytorch is an open-source machine learning library based on Torch library. Pytorch is a Tensor computing library and also 
  a deep learning library with automatic differentiation system. In can be used to build state-of-the-art Neural Networks which are used in
  self-driving cars.\par
  The crop detection and crop disease detection is done using a Convolutional Neural Network(CNN) built using ResNet50 as a feature extractor.
  The current models have an accuracy of 92\% and 90\%.

  \item[NumPy: ] NumPy is a numerical library that is used for accelerated tensor operations. This help is speeding up the computation by 
  taking advantage of SIMD instructions on modern CPUs and GPUs. This library is useful for preprocessing data before classification.

  \item[Matplotlib: ] Matplotlib is a plotting library for python. This is useful for EDA and performing error analysis while training 
  and testing the AI models.

  \item[PIL: ] Python Image Library is an free and open-source that supports opening, modifying, saving images. This is used to convert data
  sent over internet as bytes to images that can then be used in the AI models for inference. PIL is also used to load images from disk for
  training the models.

  \item[Colab: ] Google Colab is a free cloud service which helps in developing deep learning applications using popular libraries such 
  as Keras, TensorFlow, PyTorch. In runs on Google servers with GPU or TPU acceleration which helps in reducing the training period for 
  the AI model. Giving team, the ability to iterate and improve quickly/

  \item[Flask and Docker: ] Flask is a light-weight web framework for python. This can be used to run the inference server. Flask is ran
  inside a docker container. Docker is a set of platform as a service products that uses OS-level virtualization to deliver software in 
  packages called containers. These container can easily be scaled up or down to automatically adjust for usage and network traffic.

  \item[Android Studio: ] Android Studio is the official IDE for developing native Android Apps. Android Studio provides a unified  
  environment where you can build apps for Android phones, tablets, Android Wear, Android TV, and Android Auto. Structured code modules 
  allow you to  divide your project into units of functionality that you can independently build, test, and debug.\par
  Android's Location library can be used to obtain user's location. Google Maps can we used to display user's location and provide 
  additional information.

  \item[Xcode: ] Xcode is the official IDE for developing native iOS apps. Xcode provides a unified environment for developing native apps 
  for all apple devices and services. Swift programming language is used to write apps in Xcode. CoreLocation framework can be used to 
  obtain user's location. MapKit (Apple Maps) can be used to display user's location and provide additional information.\par

  \item Apps can be localized to make them be available in multiple language, so more people can use the app without english being a 
  hindrance.

  \item[Firebase: ] Firebase is a BaaS (Backend as a Service) provided by Google. Firebase is scalable, distributed and secure by design. It is especially
  geared towards business apps, with the intention of helping businesses grow their user bases and increase their profits through their mobile apps.\par
  Firebase provides authentication(Auth), Database (Firestore), Cloud Storage (Storage), Analytics as a service.
\end{description}

\subsection{Hardware Specifications}
\begin{description}
  \item[User Device Requirements: ]
  \item Any Android phone running Android Oreo or above
  \item iPhone or iPad running iOS 13 or above
  \item 
  
  \item [Minimum Server Requirements: ]
  \item Processor: Any x86 CPU with clock sspeed of 2.5GHz and above
  \item RAM: 2GB
  \item Storage: Atleast 10GB
\end{description}
Above mentioned server requirements are minimum and need to be automatically scaled depending on usage and traffic

\end{document}